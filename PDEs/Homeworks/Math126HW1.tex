\documentclass{article}
\usepackage{amsmath}
\usepackage{amssymb}
\usepackage{tikz}
\usepackage{quiver}
\usepackage{mathtools}
\usepackage{geometry}
\usepackage{amsthm}
\usepackage{leftindex}


\newcommand{\R}{\mathbb{R}} 
\newcommand{\N}{\mathbb{N}}
\newtheorem{definition}{Definition}[section]
\newtheorem{exmp}{Example}[section]
\newtheorem{prop}{Proposition}[section]
\newtheorem{rmk}{Remark}[section]
\newtheorem{lemma}{Lemma}[section]

\newgeometry{
    top=0.75in,
    bottom=0.75in,
    outer=0.75in,
    inner=0.75in
}

\title{Math 126 Homework 1}
\author{Aniruddh V.}
\date{September 2023}

\begin{document}

\maketitle

\section{Problem 1}
\textbf{Solution } Suppose $f_n$ converges to $f$ uniformly in $C([0, 1])$. Then $ \| f_n - f\| \rightarrow 0$ as $n \rightarrow \infty$, so:

\begin{align*}
    \left | \int_0^1 f_n(x)dx - \int_0^1 f(x)dx      \right | & \leq \int_0^1 | f_n(x) - f(x)| dx \\
    & \leq \int_0^1 \| f_n(x) - f(x) \| dx \\
    &= \| f_n(x) - f(x) \| 
\end{align*} which goes to zero as $n \rightarrow \infty$, so the order of limit and integration can be swapped if $f_n \rightarrow f$ uniformly. For a counterexample 
when $f_n \rightarrow f$ pointwise, consider the sequence of piecewise continous functions on $[0, 1]$

\begin{equation*}
    f_n = \begin{cases}
            n^2 x & 0  \leq x \leq \frac{1}{2n} \\

            -n^2 x + n  & \frac{1}{2n} < x \leq \frac{1}{n} \\

            0 &  \frac{1}{n} < x \leq 1
        
        \end{cases}
\end{equation*}

Then $f_n \rightarrow 0$ pointwise on $[0, 1]$. But for all $n$:

\[\int_0^1 f_n(x)dx = \int_0^{1/2n} n^2x dx + \int_{1/2n}^{1/n} -n^2 x + n dx = \frac{1}{8}\]

But 
\[\int_0^1 f(x) dx = \int_0^1 0dx = 0\] and thus
\[ \lim_{n \rightarrow \infty} \int_0^1 f_n(x) dx = \frac{1}{8} \neq 0 = \int_0^1 f(x) dx = \int_0^1 \lim_{n\rightarrow \infty} f_n (x) dx\] 
and thus the order of limits and integration cannot be swapped when $f_n \rightarrow f$ pointwise.

\section{Problem 2}  
\textbf{Solution } Clearly $C_0(\R^n) \subset C(\R ^n)$ is a subsapce, since if $f_1, f_2 \in C_0(\R^n)$ and $c_1, c_2 \in \R$. Then
\[ \lim_{|x| \rightarrow \infty}c_1 f_1 + c_2 f_2  = c_1 \lim_{|x| \rightarrow \infty} f_1  + c_2 \lim_{|x| \rightarrow \infty} f_2 = 0\] so $C_0(\R^n)$
is closed under linear combinations. Obviously, the zero function is in $C_0(\R^n)$. All that remains to show is that the uniform limit of a sequence $f_n$ in $C_0(\R^n)$
is also in $C_0(\R^n)$. 

To this end, first note that if a sequence $f_n \rightarrow f$ uniformly with $f_n$ continuous, then $f$ is also continuous. Thus all that remains is to show that
\( \lim_{|x| \rightarrow \infty} f(x) = 0  \). Fix $\epsilon > 0$. Since $f_n(x) \rightarrow 0$, for each $n$ there is a corresponding constant $a_n$ such that if $|x| > a_n$, then
$|f_n(x)| < \epsilon /2 $. Let $a = \sup \{a_1, a_2, \ldots\}$.  

Next, since $f_n \rightarrow f$ uniformly, there is $N \in \N$ such that for all $n > N$, $ \| f - f_n \| < \epsilon/2$. So for all $ x > a$

\begin{align*}
    | f(x) | &= | f(x) - f_n(x) + f_n(x) | \\
    & \leq | f(x) - f_n(x) | + |f_n(x) | \\
    & \leq \frac{\epsilon}{2} + \frac{\epsilon}{2} \\
    & \leq \epsilon
\end{align*} 
So for any $\epsilon$, there is a constant $a$ such that $x > a \rightarrow f(x) < \epsilon$, so 
\[ \lim_{n \rightarrow \infty} f(x) = 0\] and $C_0(\R^n)$ is a closed subspace of $C(\R^n)$. 

\section{Problem 3}

\textbf{Solution} Let $u: [0, T  ) \to \R^n$ be a solution of the ODE 
\begin{equation*}
    \begin{cases}
        u' = F(u) \\
        u(0) = u_0
    \end{cases}
\end{equation*}
 where $F : \R^n \to R^n$ is a globally Lipschitz function with constant Lipschitz constant $L$. Then, for $t \in [0, T)$,  by the Fundamental Theorem of Calculus:
\begin{align*}
u(t) - u(0) = \int_0^t u'(s) ds &= \int_0^t F(u(s))ds \\
&= \int_0^t F(u(s)) + F(u(0)) - F(u(0)) ds \\
&= F(u_0)t + \int_0^t F(u(s) - F(u(0))) ds \\
\end{align*} Taking the absolute value of both sides and using the Lipschitz condition gives:
\[|u(t) - u_0| \leq T |F(u_0)| + L \int_0^t |u(s) - u_0  |ds \] 

By Gronwall's inequality, the integral in the right hand side of the above expression is bounded, and thus $u$ is bounded on $[0,  T)$. Since $T$ was arbitrary, the solution $u$ is bounded for all $T < \infty$, and thus not maximal. Thus $u$ exists for all time. 

\end{document}