\chapter{Groups}

\section{Monoids}

Let $S$ be a set. A mapping $f: S \times S \to S$ is called a law of composition. If $x, y \in S$, the evaluation $f(x, y)$ is called the product of $x$ and $y$, and is denoted $xy$.
If $x, y, z \in S$ are three different elements, we may form their product in two different ways, namely $(xy)z$ and $x(yz)$. If $x(yz) = (xy)z$, then the law of composition is 
\textbf{assosciative}, and the parantheses can be ommitted. 

An element $e \in S$ is said to be an identity element if for all $x \in S$, $ex = xe = x$. An identity element is unique, since if $e, e' \in S$ are both identities, then
$e = ee' = e'$.

\begin{definition} A set $G$, equipped with an assosciative law of composition, and an identity element, is called a \textbf{monoid} \end{definition}
\begin{definition} Given a monoid $G$, a \textbf{submonoid} $H \subset G$ is a subset containing the identity, and that is closed under the composition law of $G$. That is,
    if $x, y \in H$, then $xy, \in H$ as well.
\end{definition}

\begin{exmp}
    The set $\mathbb{N}$ of natural numbers form a monoid, where the law of composition is addition, with identity element $0$
\end{exmp}

\section{Groups}

\begin{definition} A \textbf{group} $G$ is a monoid such that for every element $x \in G$, there is an element $y \in G$ such that $xy = yx = e$. The element $y$ is called the 
    \textbf{inverse} of $x$, denoted $x^{-1}$. 
\end{definition}

An inverse of an element must be unique. To show this, suppose $y, y' \in G$ are both inverses of $x$. Then 
\[ y = ye = y(xy') = (yx)y' = ey' = y'\]

\begin{exmp}
    Let $G$ be a group, $S$ a non-empty set. Then the $M(S, G) = \{ f: S \to G  \}$ is a group, where if $f, g \in M(S, G)$, the product $fg$ is defined as $(fg)(x) = f(x)g(x)$,
    and inverses such that $f^{-1}(x) = f(x) ^{-1}$. Then $M(S, G)$ is a group. If $G$ is abelian, then so is $M(S, G)$
 \end{exmp}

 \begin{exmp}
    The set of rational numbers $\mathbb{Q}$ is a group under addition. The set of non-zero rationals $\mathbb{Q} \setminus \{0\}$ is a group under multiplication. Similar statements 
    hold for the real numbers $\mathbb{R}$ and the complex numbers.  
 \end{exmp}

 \begin{definition}
    A group $G$ is said to be \textbf{cyclic} if there is some element $a \in G$ such that for every $g \in G$, there is an integer $n$ so that $g = a^n$
 \end{definition}

 \begin{exmp}
    The set of integers $\mathbb{Z}$ is cyclic, with generator $1$, or alternatively, generator $-1$
 \end{exmp}

 \begin{exmp}
    Fix a positive integer $n$. Then, the $n$-th roots of unity in the complex numbers form a cyclic group of order $n$. A generator for this group is a complex number of 
    the form $\exp(2\pi ir/n)$, with $\gcd(r, n) = 1$
 \end{exmp}

 \begin{definition}
    Let $G_1, G_2$ be groups. The \textbf{direct product} of $G_1$ and $G_2$, denoted $G_1 \times G_2$ is the set of all pairs $(x_1, x_2)$, with $x_i \in G_i$. We define the law 
    of composition componentwise - $(x_1, x_2)(y_1, y_2) = (x_1y_1, x_2y_2)$. The identity of this group is $(e_1, e_2)$, where $e_i \in G_i$ is the identity of each respective group. 
 \end{definition}

 The idea of a direct product of groups can be extended to the product of a family of $n$ groups $G_i$, with $\prod_{i=1}^n G_i = G_1 \times G_2 \times \cdots \times G_n$, with 
 identity $(e_1, e_2, \ldots, e_n)$.

 \begin{definition}
    Let $G$ be a group. A \textbf{subgroup} $H$ of $G$ is a subset of $G$ containing the identity element, and such that $H$ is closed under the law of composition and taking inverses.
    A subgroup is \textbf{trivial} if it contains the identity element alone
 \end{definition}

 \begin{definition}
    Let $G$ be a group, and $S$ be a subset of $G$. Then $S$ \textbf{generates} $G$, or $S$ is the \textbf{set of generators} of $G$, if every element in $G$ can be expressed
    as a product of elements of $S$ or inverses of elements of $S$, that is a product $x_1 x_2 \cdots x_n$, where each $x_i$ or $x_i^{-1}$ is in $S$ 
 \end{definition}

 Clearly if $S$ generates $G$, it is a subgroup of $G$. Furthermore, $S$ is the smallest subgroup of $G$ containing $S$. If $G$ is generated by $S$, we write $G = \langle S \rangle$

 \begin{exmp}
    There are two non-abelian groups of order $8$. One is the \textbf{group of symmetries of the square}, generated by two elements $\sigma, \tau$ such that 
    $\sigma^4 = \tau^2 = e$ and $\tau \sigma \tau^{-1} = \sigma^3 $ \\
    The other is the \textbf{quaternion group}, generated by two elements $i, j$, such that if $k = ij$ and $m = i^2$, then $i^4 = j^4 = k^4 = e$, $i^2 = j^2 = k^2 = m$, and $ij = mji$
 \end{exmp}
