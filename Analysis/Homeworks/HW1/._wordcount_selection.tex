\section{Problem 20}

\textbf{Solution } Suppose $\rho \in \text{Rel}_2(X)$ and $\rho \implies \leq$. This means that if $\rho(x, x')$, then $\langle x \rvert \subseteq \langle x' \rvert$. Clearly 
$\langle x \rvert \subseteq \langle x \rvert$, so $\rho(x, x)$ must hold, and $\rho$ must be reflexive. Next, if $\langle x \rvert \subseteq \langle x' \rvert$ and $\langle x' \rvert \subseteq \langle x'' \rvert$,
then $\langle x \rvert \subseteq \langle x'' \rvert$, so $\rho(x, x'), \rho(x', x'' ) \implies \rho(x, x'')$, and so $\rho$ is transitive. Finally, $\rho$ must be weakly antisymmetric, since if 
$\langle x \rvert \subseteq \langle x' \rvert$ and $\langle x' \rvert \subseteq \langle x \rvert$, then $\langle x \rvert = \langle x' \rvert$. Thus $\rho$ must be an order relation.


\section{Problem 21}

\textbf{Solution } Suppose $\rho \in \text{Rel}_2(X)$ and $\leq \implies \rho$. This means that if $\langle x \rvert \subseteq \langle x' \rvert$, then $\rho(x, x')$ holds. But $\subseteq$ is an
order relation, so then $\rho$ must also be an order relation.