\documentclass{article}
\usepackage{amsmath}
\usepackage{amssymb}
\usepackage{tikz}
\usepackage{quiver}
\usepackage{mathtools}
\usepackage{geometry}
\usepackage{mathrsfs}

\newcommand{\id}{\text{id}}

\newgeometry{
    top=0.75in,
    bottom=0.75in,
    outer=0.75in,
    inner=0.75in
}

\title{Math H104 Homework 3}
\author{Aniruddh V.}
\date{September 2023}

\begin{document}

\maketitle

\section{Exercise 16}

\textbf{Solution } We have the following commutative diagram: 

% https://q.uiver.app/#q=WzAsNixbNCwwLCJYXzEiXSxbMiwwLCJZXzEiXSxbMCwwLCJaXzEiXSxbMCwyLCJaXzIiXSxbMiwyLCJZXzIiXSxbNCwyLCJYXzIiXSxbMCw1LCJcXGdhbW1hIl0sWzUsNCwiXFxjaGlfMiJdLFs0LDMsIlxccGhpXzIiLDAseyJzdHlsZSI6eyJ0YWlsIjp7Im5hbWUiOiJtb25vIn19fV0sWzIsMywiXFxhbHBoYSAiLDJdLFswLDEsIlxcY2hpXzEiLDIseyJzdHlsZSI6eyJoZWFkIjp7Im5hbWUiOiJlcGkifX19XSxbMSwyLCJcXHBoaV8xIiwyXV0=
\[\begin{tikzcd}
	{Z_1} && {Y_1} && {X_1} \\
	\\
	{Z_2} && {Y_2} && {X_2}
	\arrow["\gamma", from=1-5, to=3-5]
	\arrow["{\chi_2}", from=3-5, to=3-3]
	\arrow["{\phi_2}", tail, from=3-3, to=3-1]
	\arrow["{\alpha }"', from=1-1, to=3-1]
	\arrow["{\chi_1}"', two heads, from=1-5, to=1-3]
	\arrow["{\phi_1}"', from=1-3, to=1-1]
\end{tikzcd}\]

Since $\chi_1$ is injective, it has a left inverse $\chi_1^{-1}$ such that $\chi_1^{-1} \circ \chi_1 = id_{X_1}$. Similarly, since $\phi_2$ is surjective, it has a right inverse $\phi_2^{-1}$
such that $\phi_2 \circ \phi_2^{-1} = id_{Z_2}$. 

Now suppose there are $\beta_1$ and $\beta_2$ such that the diagram below commutes. 

% https://q.uiver.app/#q=WzAsNixbNCwwLCJYXzEiXSxbMiwwLCJZXzEiXSxbMCwwLCJaXzEiXSxbMCwyLCJaXzIiXSxbMiwyLCJZXzIiXSxbNCwyLCJYXzIiXSxbMCw1LCJcXGdhbW1hIl0sWzUsNCwiXFxjaGlfMiJdLFs0LDMsIlxccGhpXzIiLDAseyJzdHlsZSI6eyJ0YWlsIjp7Im5hbWUiOiJtb25vIn19fV0sWzIsMywiXFxhbHBoYSAiLDJdLFswLDEsIlxcY2hpXzEiLDIseyJzdHlsZSI6eyJoZWFkIjp7Im5hbWUiOiJlcGkifX19XSxbMSwyLCJcXHBoaV8xIiwyXSxbMSw0LCJcXGJldGFfMSIsMSx7ImN1cnZlIjozLCJzdHlsZSI6eyJib2R5Ijp7Im5hbWUiOiJkYXNoZWQifX19XSxbMSw0LCJcXGJldGFfMiIsMSx7ImN1cnZlIjotMywic3R5bGUiOnsiYm9keSI6eyJuYW1lIjoiZGFzaGVkIn19fV1d
\[\begin{tikzcd}
	{Z_1} && {Y_1} && {X_1} \\
	\\
	{Z_2} && {Y_2} && {X_2}
	\arrow["\gamma", from=1-5, to=3-5]
	\arrow["{\chi_2}", from=3-5, to=3-3]
	\arrow["{\phi_2}", tail, from=3-3, to=3-1]
	\arrow["{\alpha }"', from=1-1, to=3-1]
	\arrow["{\chi_1}"', two heads, from=1-5, to=1-3]
	\arrow["{\phi_1}"', from=1-3, to=1-1]
	\arrow["{\beta_1}"{description}, curve={height=18pt}, dashed, from=1-3, to=3-3]
	\arrow["{\beta_2}"{description}, curve={height=-18pt}, dashed, from=1-3, to=3-3]
\end{tikzcd}\]

With the inverses added in, the diagram looks as follows:

% https://q.uiver.app/#q=WzAsNixbNCwwLCJYXzEiXSxbMiwwLCJZXzEiXSxbMCwwLCJaXzEiXSxbMCwyLCJaXzIiXSxbMiwyLCJZXzIiXSxbNCwyLCJYXzIiXSxbMCw1LCJcXGdhbW1hIl0sWzUsNCwiXFxjaGlfMiJdLFs0LDMsIlxccGhpXzIiLDAseyJzdHlsZSI6eyJ0YWlsIjp7Im5hbWUiOiJtb25vIn19fV0sWzIsMywiXFxhbHBoYSAiLDJdLFsxLDIsIlxccGhpXzEiLDJdLFsxLDQsIlxcYmV0YV8xIiwxLHsiY3VydmUiOjMsInN0eWxlIjp7ImJvZHkiOnsibmFtZSI6ImRhc2hlZCJ9fX1dLFsxLDQsIlxcYmV0YV8yIiwxLHsiY3VydmUiOi0zLCJzdHlsZSI6eyJib2R5Ijp7Im5hbWUiOiJkYXNoZWQifX19XSxbMyw0LCJcXHBoaV8yXnstMX0iLDIseyJjdXJ2ZSI6Mywic3R5bGUiOnsiYm9keSI6eyJuYW1lIjoiZGFzaGVkIn19fV0sWzAsMSwiXFxjaGlfMSIsMix7InN0eWxlIjp7ImhlYWQiOnsibmFtZSI6ImVwaSJ9fX1dLFsxLDAsIlxcY2hpXzFeey0xfSIsMCx7ImN1cnZlIjotMywic3R5bGUiOnsiYm9keSI6eyJuYW1lIjoiZGFzaGVkIn19fV1d
\[\begin{tikzcd}
	{Z_1} && {Y_1} && {X_1} \\
	\\
	{Z_2} && {Y_2} && {X_2}
	\arrow["\gamma", from=1-5, to=3-5]
	\arrow["{\chi_2}", from=3-5, to=3-3]
	\arrow["{\phi_2}", tail, from=3-3, to=3-1]
	\arrow["{\alpha }"', from=1-1, to=3-1]
	\arrow["{\phi_1}"', from=1-3, to=1-1]
	\arrow["{\beta_1}"{description}, curve={height=18pt}, dashed, from=1-3, to=3-3]
	\arrow["{\beta_2}"{description}, curve={height=-18pt}, dashed, from=1-3, to=3-3]
	\arrow["{\phi_2^{-1}}"', curve={height=18pt}, dashed, from=3-1, to=3-3]
	\arrow["{\chi_1}"', two heads, from=1-5, to=1-3]
	\arrow["{\chi_1^{-1}}", curve={height=-18pt}, dashed, from=1-3, to=1-5]
\end{tikzcd}\]
From these diagrams we can read off the following relations: $\alpha \circ \phi_1 = \phi_2 \circ \beta_1$ and $\beta_2 \circ \chi_1 = \chi_2 \circ \gamma$. Now using applying the left and right inverses
gives $\beta_1 = \phi_2^{-1} \circ \alpha \circ \phi_1$ and $\beta_2 = \chi_2 \circ \gamma \circ \chi_1^{-1}$. But by the commutativity of the first diagram, we have 
$\alpha \circ \phi_1 \circ \chi_1 = \phi_2 \circ \chi_2 \circ \gamma$, and applying inverses gives $ \phi_2^{-1} \circ \alpha \circ \phi_1 = \chi_2 \circ \gamma \circ \chi_1^{-1} $. But 
then we have 
\[ \beta_1 = \phi_2^{-1} \circ \alpha \circ \phi_1 = \chi_2 \circ \gamma \circ \chi_1^{-1} = \beta_2 \] so $\beta_1 = \beta_2 \coloneqq \beta$ is unique, as desired.

\section{Exercise 27}

\textbf{Solution } The functions from $\mathscr{P}\mathscr{P}X \to \mathscr{P}\mathscr{P}Y$ are $f_{**}, f_{!*}, f_{*!}, f_{!!}$ and $f^{**}$. 

\section{Exercise 33}

\textbf{Solution } The properties that are inherited by $\phi^* \rho$ are reflexivity, transitivity, and symmetricity

\section{Exercise 35}

\textbf{Solution } We want to show the following diagram commutes: 

% https://q.uiver.app/#q=WzAsNCxbMCwwLCJcXG1hdGhzY3J7UH1YJyJdLFswLDIsIlxcbWF0aHNjcntQfVgiXSxbMiwwLCJcXG1hdGhzY3J7UH1ZJyJdLFsyLDIsIlxcbWF0aHNjcntQfVkiXSxbMCwyLCJSIl0sWzMsMiwiZ18qIiwyXSxbMSwwLCJmXyoiXSxbMSwzLCJSIiwyXSxbNCw1LCJcXHN1cHNldGVxIiwwLHsiY3VydmUiOjIsInNob3J0ZW4iOnsic291cmNlIjoyMCwidGFyZ2V0IjoyMH0sImNvbG91ciI6WzAsNjAsNjBdfSxbMCw2MCw2MCwxXV1d
\[\begin{tikzcd}
	{\mathscr{P}X'} && {\mathscr{P}Y'} \\
	\\
	{\mathscr{P}X} && {\mathscr{P}Y}
	\arrow[""{name=0, anchor=center, inner sep=0}, "R", from=1-1, to=1-3]
	\arrow[""{name=1, anchor=center, inner sep=0}, "{g_*}"', from=3-3, to=1-3]
	\arrow["{f_*}", from=3-1, to=1-1]
	\arrow["R"', from=3-1, to=3-3]
	\arrow["\supseteq", color={rgb,255:red,214;green,92;blue,92}, curve={height=12pt}, shorten <=8pt, shorten >=8pt, Rightarrow, from=0, to=1]
\end{tikzcd}\]

Equivalently, we must show that for any set $A \in \mathscr{P}X$, $g_* R A \subseteq R f_* A$. The set $g_* RA$ is given by

\[ g_* RA = \{g(y) : \rho(x, y) \forall x \in A\} \] and the set $R f_* A$ is given by 
\[ R f_* A = \{ y' : \rho ' (f(x), y') \forall x \in A  \}  \] But since $(f, g)$ is a morphism of binary relations, we have $(f, g)^* \rho ' = \rho ' \circ (f, g)$, so 
if $\rho(x, y)$ holds, then so does $\rho ' (f(x), g(y))$. Thus, if $y' = g(y) \in g_* RA$, then $y' \in R f_* A$, and we have  $ g_* R \subseteq Rf_* $.

Now consider the following diagram: 

% https://q.uiver.app/#q=WzAsNCxbMCwwLCJcXG1hdGhzY3J7UH1YJyJdLFswLDIsIlxcbWF0aHNjcntQfVgiXSxbMiwwLCJcXG1hdGhzY3J7UH1ZJyJdLFsyLDIsIlxcbWF0aHNjcntQfVkiXSxbMywyLCJnXyoiLDJdLFsxLDAsImZfKiJdLFsyLDAsIkwiLDJdLFszLDEsIkwiXSxbNSw2LCJcXHN1YnNldGVxIiwwLHsiY3VydmUiOjIsInNob3J0ZW4iOnsic291cmNlIjoyMCwidGFyZ2V0IjoyMH0sImNvbG91ciI6WzAsNjAsNjBdfSxbMCw2MCw2MCwxXV1d
\[\begin{tikzcd}
	{\mathscr{P}X'} && {\mathscr{P}Y'} \\
	\\
	{\mathscr{P}X} && {\mathscr{P}Y}
	\arrow["{g_*}"', from=3-3, to=1-3]
	\arrow[""{name=0, anchor=center, inner sep=0}, "{f_*}", from=3-1, to=1-1]
	\arrow[""{name=1, anchor=center, inner sep=0}, "L"', from=1-3, to=1-1]
	\arrow["L", from=3-3, to=3-1]
	\arrow["\subseteq", color={rgb,255:red,214;green,92;blue,92}, curve={height=12pt}, shorten <=8pt, shorten >=8pt, Rightarrow, from=0, to=1]
\end{tikzcd}\]

We want to show that for any $B \in \mathscr{P}Y$, $f_* LB \subseteq L g* B$. The set $f_* L B$ is given by 
\[ f_* LB = \{f(x) : \rho(x, y) \forall y \in B \} \] and the set $L g_* B$ is given by
\[ Lg_* B = \{x' : \rho '(x', g(y))  \forall y \in B\}\] But $(f, g)$ is a morphism of binary relations, so $(f, g)^* \rho ' = \rho ' \circ (f, g)$, so 
if $\rho(x, y)$ holds, then so does $\rho ' (f(x), g(y))$. Thus, if $x' = f(x) \in f_* LB$, then $x' \in L g_* B$, and we have $f_* L \subseteq Lg_*$.


\section{Exercise 39}

The relations $\leq$ and $\geq$ are both antisymmetric transitive reflexive relations, in other words they are order relations. z

\section{Exercise 40}

\textbf{Solution } Suppose $\rho \in \text{Rel}_2(X)$ and $\rho \implies \leq$. This means that if $\rho(x, x')$, then $\langle x \rvert \subseteq \langle x' \rvert$. Clearly 
$\langle x \rvert \subseteq \langle x \rvert$, so $\rho(x, x)$ must hold, and $\rho$ must be reflexive. Next, if $\langle x \rvert \subseteq \langle x' \rvert$ and $\langle x' \rvert \subseteq \langle x'' \rvert$,
then $\langle x \rvert \subseteq \langle x'' \rvert$, so $\rho(x, x'), \rho(x', x'' ) \implies \rho(x, x'')$, and so $\rho$ is transitive. Finally, $\rho$ must be weakly antisymmetric, since if 
$\langle x \rvert \subseteq \langle x' \rvert$ and $\langle x' \rvert \subseteq \langle x \rvert$, then $\langle x \rvert = \langle x' \rvert$. Thus $\rho$ must be an order relation.


\section{Exercise 41}

\textbf{Solution } Suppose $\rho \in \text{Rel}_2(X)$ and $\leq \implies \rho$. This means that if $\langle x \rvert \subseteq \langle x' \rvert$, then $\rho(x, x')$ holds. But $\subseteq$ is an
order relation, so then $\rho$ must also be an order relation.

\section{Exercise 43} \textbf{Solution } Let $y \in \cap R_* \mathscr{A}$. Then for all $A \in \mathscr{A}$, we have $y \in R_* A$, so clearly there is $x \in X$ such that 
$\rho(x, y)$ holds for each $x \in A \in \mathcal{A}$ so clearly $x \in \cup \mathscr{A}$. Thus $y \in R(\cup \mathscr{A})$. On the other hand, suppose $y \in R(\cup \mathscr{A}) $.
Then for all $x \in A \in \mathscr{A}$, the relation $\rho(x, y)$ holds, so for all $A \in \mathscr{A}$, $y \in R_* A$, and thus $y \in \cap R_* \mathscr{A}$. 

\section{Exercise 44}
\textbf{a. } \textbf{Solution } By definition, we have 
\[ RX = \{y \in Y | \forall x \in X,  \rho(x, y)\} \]
But by definition, if $\rho(x, y)$ holds for all $x \in X$, then $y$ is a terminal element of $Y$. Thus we have
\[ RX = \{y \in Y | \text{ $y$ is a terminal element of $Y$ }\} \] Similarly, if $\rho(x, y)$ holds for all $y \in Y$, then $x$ is an initial element of $X$, so we have 
\[  LY = \{x \in X | \text{ $x$ is an initial element of $X$}\} \]

\textbf{b. } \textbf{Solution } Suppose $\xi \in X $ is a supremum of $X$. Then we have $ RX = \lvert \xi \rangle $, which means $\xi$ is the least element of a preordered set
$(X, \geq)$. Now suppose $\xi$ is a smallest element of $(X, \geq)$. Then we have $ RX = \lvert \xi \rangle $, so $\xi \in X $
 is a supremum of $X$

 Suppose $\nu \in Y $ is a supremum of $Y$. Then we have $ LY = \langle \nu \rvert $, which means $\nu$ is the least element of a preordered set
$(Y, \leq)$. Now suppose $\nu$ is a smallest element of $(Y, \leq)$. Then we have $ LY = \langle \nu \rvert $, so $\nu \in Y $
 is a supremum of $Y$
\end{document}