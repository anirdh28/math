\section{Introduction}
In a sense, this is what the language we have developed so far has been building up to. In fact, when MacLane and Eillenberg first started developing category theory, their motivation 
was to rigorously define what it meant for a mapping to be "natural", and came up with our modern notion of natural transformations. However, to rigorously define natural transformations,
they first needed to define functors, and to define functors they needed the idea of a category. We begin with a motivating idea from linear algebra on the distinction between natural
and non-natural functors

\section{A motivating example}
Consider a finite dimensional vector space $V$ over an arbitrary field $k$. To each space $V$, one can associate the dual space $V*$, defined as 
\[ V^* = \{ f : V \to k | f \text{ linear} \}  \]
and the double dual $V**$ defined by
\[V^{**} = \{f : V^* \to k | f \text{ linear}\}\] It is a well known fact that both $V^*$ and $V^{**}$ are isomorphic to $V$, but there is a distinction between them. To construct the
isomorphism $V \to V^*$, one chooses a basis for $V$ and constructs the corresponding dual basis for $V^*$. So the isomorphism depends on the basis chosen for $V$. On the other hand, 
the isomorphism $V \to V^{**}$ is created by noticing that the map $v \mapsto E_v$ such that $E_v(\phi) = \phi(v)$ is injective. This map required no choice of basis, and thus feels more
"natural" in the sense that less choices needed to made for the isomorphism. 

\section{Definitions}

\begin{definition}
    Intuitively, we want to talk about morphisms between functors. Fix two categories $C$ and $D$, and let $S: C \to D$ and $T: tC \to D$ be functors. 
A natural transformation $\tau: S \dotarrow T$ is a function which assigns to each object $c$ of $C$ an arrow $\tau_c = \tau c : Sc \to Tc$ of $D$ such that 
every arrow $f: c \to c'$ in $C$ yields a commutative diagram 

% https://q.uiver.app/#q=WzAsNixbMCwwLCJjIl0sWzAsMiwiYyciXSxbMiwwLCJTYyJdLFsyLDIsIlNjJyJdLFs0LDIsIlRjJyJdLFs0LDAsIlRjIl0sWzAsMSwiZiJdLFsyLDMsIlNmIiwyXSxbMiw1LCJcXHRhdSBjIl0sWzUsNCwiVGYiXSxbMyw0LCJcXHRhdSBjJyIsMl1d
\[\begin{tikzcd}
	c && Sc && Tc \\
	\\
	{c'} && {Sc'} && {Tc'}
	\arrow["f", from=1-1, to=3-1]
	\arrow["Sf"', from=1-3, to=3-3]
	\arrow["{\tau c}", from=1-3, to=1-5]
	\arrow["Tf", from=1-5, to=3-5]
	\arrow["{\tau c'}"', from=3-3, to=3-5]
\end{tikzcd}\]

When such a diagram exists, we say that $\tau_c : Sc \to Tc$ is natural in $c$. The functions $\tau a, \tau b, \tau c$ are called the components of the natural transformation $\tau$.
\end{definition}



\begin{definition}
    A natural transformation $\tau: C \dotarrow D$ with every component $\tau c \in D$ invertible is called a natural isomorphism, denoted $\tau: S \cong T$. The inverses
    $(\tau c)^{-1} \in D$ are the components of the natural isomorphism $\tau^{-1} : T \dotarrow S$.  
\end{definition}
