\documentclass{article}
\usepackage{amsmath}
\usepackage{amssymb}
\usepackage{tikz}
\usepackage{quiver}
\usepackage{mathtools}
\usepackage{geometry}
\usepackage{amsthm}
\usepackage{leftindex}


\newcommand{\R}{\mathbb{R}} 
\newcommand{\N}{\mathbb{N}}
\newtheorem{definition}{Definition}[section]
\newtheorem{exmp}{Example}[section]
\newtheorem{prop}{Proposition}[section]
\newtheorem{rmk}{Remark}[section]
\newtheorem{lemma}{Lemma}[section]

\newgeometry{
    top=0.75in,
    bottom=0.75in,
    outer=0.75in,
    inner=0.75in
}

\title{Math 174 Homework 3}
\author{Aniruddh V.}
\date{September 2023}

\begin{document}



\maketitle

\section{Excercise 1.3.iii}
\textbf{Solution } Consider a category $C$, with four objects $a, b, c$ and $d$, with morphisms $f: a \to b$ and $g: c \to d$, and a category $D$ with objects $x, y, z$, and morphims 
$x \to y$, $y \to z$, and $z \to x$. Consider the functor $F : C \to D$ which maps $F(a) = x$, $F(b) = F(c) = y$, and $F(d) = z$. Since categories are closed under composiition of morphisms,
and there is no morphism $a \to d$, the image of $F$ cannot be a category.

\section{Exercise 1.3.iv}
\textbf{Solution } Let $C$ be a locally samll catgeory. Let the functor $Mor(c, -) : C \to \text{Set}$ have the following assignments. Given an object $x \in C$, $x \mapsto Mor(c, x)$. 
Next, given a morphism $f: x \to y$ in $C$, $Mor(c, -)$ acting on the morphism gives a new map $f_* : Mor(c, x) \to Mor(c, y)$. We now check the Functoriality axioms. First, $F(id_c) = id_{F(c)}$. 
So given the identity morphism $id_x : x \to x$, $Mor(c,  -)$ maps $id_x$ to the map $Mor(c, x) \to Mor(c,x )$. That is $id_x \mapsto id_{Mor(c, x)}$, and the first axiom of functoriality is
satisfied. Next, given two morphsims $a: x \to y$ and $b: y \to z$ in $C$, we want to show that $F(ab) = F(a)F(b)$. Indeed, $ab$ is a morphism from $x \to z$, and so $F(ab)$ must be a morphism from
$Mor(c, x) \to Mor(c,z )$. But $F(a)F(b)$ is the morphism $Mor(c, x) \to Mor(c, y) \to Mor(c,z )$, which when composed together gives the same as $F(ab)$. Thus $F(ab)
 = F(a)F(b)$, and so the axioms of functoriality are satisfied for $Mor(c, -)$

 Now consider the contravariant functor $Mor(-, c)$. This behaves similarly to $Mor(c, -)$, except now a morphism $f: x \to y$ in $C$ is mapped to a morphism $f^* : Mor(y,c ) \to Mor(y. c)$. 
 The idenity law of functoriality remains the same as for $Mor(c, -)$, now the only thing left to be checked is $F(gf) = F(f)F(g)$ for morphisms $f: x \to y$ and $g: y \to z$. But $gf$ is a morphism from $x \to z$
 and since $F(f)F(g)$ also maps $z \to x$, we have equality. 

 Thus $Mor(c, -)$ and $Mor(-, c)$ are both functors

 \section{Exercise 1.4.i}
 \textbf{Solution} Let $\alpha : C \implies D$. Then for all morphisms $f: x \to y$ in $C$, the following diagram commutes

 % https://q.uiver.app/#q=WzAsNCxbMCwwLCJGeCJdLFsyLDAsIkd4Il0sWzAsMiwiRnkiXSxbMiwyLCJHeSJdLFswLDIsIkZmIiwyXSxbMSwzLCJHZiJdLFswLDEsIlxcYWxwaGFfeCJdLFsyLDMsIlxcYWxwaGFfeSIsMl1d
\[\begin{tikzcd}
	Fx && Gx \\
	\\
	Fy && Gy
	\arrow["Ff"', from=1-1, to=3-1]
	\arrow["Gf", from=1-3, to=3-3]
	\arrow["{\alpha_x}", from=1-1, to=1-3]
	\arrow["{\alpha_y}"', from=3-1, to=3-3]
\end{tikzcd}\] Further, $\alpha_x$ and $\alpha_y$ are isomorphisms. This tells us that $Gf \circ \alpha_x = \alpha_y \circ Ff$. But since $\alpha$ is an isomorphism, it has an inverse for $x$ and $y$,
so we also have $\alpha_y ^{-1} \circ Gf = Ff \circ \alpha_x$. So we have the following commutative diagram as well. 

% https://q.uiver.app/#q=WzAsNSxbMCwwLCJHeCJdLFsyLDAsIkd4Il0sWzAsMiwiR3kiXSxbMiwyLCJGeSJdLFsyLDFdLFswLDIsIkdmIiwyXSxbMSwzLCJGZiJdLFswLDEsIlxcYWxwaGFfeCBeey0xfSJdLFsyLDMsIlxcYWxwaGFfeSBeey0xfSIsMl1d
\[\begin{tikzcd}
	Gx && Gx \\
	&& {} \\
	Gy && Fy
	\arrow["Gf"', from=1-1, to=3-1]
	\arrow["Ff", from=1-3, to=3-3]
	\arrow["{\alpha_x ^{-1}}", from=1-1, to=1-3]
	\arrow["{\alpha_y ^{-1}}"', from=3-1, to=3-3]
\end{tikzcd}\] This shows $\alpha^{-1} : G \implies F$ is a natural isomorphism.


\end{document}